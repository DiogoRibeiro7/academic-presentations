\documentclass[aspectratio=169]{beamer}

% ================================================================
% Explainable AI and Model Interpretability
% Understanding and explaining machine learning models
% ================================================================

% Use the ESMAD theme
\usepackage{../theme/esmad_beamer_theme}

% Additional packages
\usepackage{subcaption}
\usepackage{multirow}

% Author information
\authorname{Diogo Ribeiro}
\authoremail{dfr@esmad.ipp.pt}
\authororcid{0009-0001-2022-7072}
\authorinstitution{ESMAD - Escola Superior de Média Arte e Design}
\authorcompany{Lead Data Scientist, Mysense.ai}

% Presentation information
\title{Explainable AI and Model Interpretability}
\subtitle{From Black Boxes to Transparent Systems}
\date{\today}

\begin{document}

% ================================================================
% Title and TOC
% ================================================================

\begin{frame}
  \titlepage
\end{frame}

\tocslide

% ================================================================
\section{Introduction}
% ================================================================

\begin{frame}{The Interpretability Crisis}
  \textbf{Modern ML models are powerful but opaque:}

  \begin{columns}
    \column{0.5\textwidth}
    \textbf{High Performance:}
    \begin{itemize}
      \item Deep neural networks
      \item Ensemble methods
      \item Complex feature interactions
    \end{itemize}

    \column{0.5\textwidth}
    \textbf{Low Interpretability:}
    \begin{itemize}
      \item Millions of parameters
      \item Non-linear transformations
      \item Difficult to explain
    \end{itemize}
  \end{columns}

  \vspace{1em}

  \begin{alertbox}{Why Does Interpretability Matter?}
    \begin{itemize}
      \item \textbf{Trust:} Users need to understand decisions
      \item \textbf{Debugging:} Identify and fix model errors
      \item \textbf{Regulation:} GDPR "right to explanation"
      \item \textbf{Fairness:} Detect and mitigate bias
      \item \textbf{Scientific insight:} Understand underlying phenomena
    \end{itemize}
  \end{alertbox}
\end{frame}

\begin{frame}{Interpretability vs. Explainability}
  \begin{definitionbox}{Interpretability}
    The degree to which a human can \textbf{understand} the cause of a decision.

    \textit{Example:} Linear regression with few features is inherently interpretable.
  \end{definitionbox}

  \vspace{0.5em}

  \begin{definitionbox}{Explainability}
    The degree to which a human can \textbf{consistently predict} the model's result.

    \textit{Example:} LIME provides explanations for black-box models.
  \end{definitionbox}

  \vspace{0.5em}

  \textbf{Key distinction:}
  \begin{itemize}
    \item \textbf{Interpretability:} Intrinsic property of the model
    \item \textbf{Explainability:} Post-hoc analysis of model behavior
  \end{itemize}
\end{frame}

\begin{frame}{The Accuracy-Interpretability Tradeoff}
  \begin{center}
    \textbf{Traditional view: Must choose between accuracy and interpretability}
  \end{center}

  \vspace{0.5em}

  \begin{columns}
    \column{0.5\textwidth}
    \textbf{Inherently Interpretable:}
    \begin{itemize}
      \item Linear/logistic regression
      \item Decision trees (shallow)
      \item Generalized additive models (GAM)
      \item Rule-based systems
    \end{itemize}

    \textcolor{ESMADBlue}{\textbf{High interpretability, Lower accuracy}}

    \column{0.5\textwidth}
    \textbf{Black Box Models:}
    \begin{itemize}
      \item Deep neural networks
      \item Random forests (large)
      \item Gradient boosting
      \item SVM with RBF kernel
    \end{itemize}

    \textcolor{ESMADAccent}{\textbf{Low interpretability, Higher accuracy}}
  \end{columns}

  \vspace{1em}

  \begin{block}{Modern Approach}
    Use explanation methods to make black-box models transparent!
  \end{block}
\end{frame}

% ================================================================
\section{Global vs. Local Explanations}
% ================================================================

\begin{frame}{Scope of Explanations}
  \begin{definitionbox}{Global Explanations}
    Describe the \textbf{overall} behavior of the model across all predictions.

    \textit{Question:} How does the model work in general?
  \end{definitionbox}

  \vspace{0.5em}

  \begin{definitionbox}{Local Explanations}
    Explain a \textbf{specific} prediction for a single instance.

    \textit{Question:} Why did the model make this particular prediction?
  \end{definitionbox}

  \vspace{0.5em}

  \begin{examplebox}{Example: Credit Scoring}
    \textbf{Global:} "Income is the most important feature overall"

    \textbf{Local:} "Applicant X was denied because their income (\$30K) is below the threshold (\$35K)"
  \end{examplebox}
\end{frame}

\begin{frame}{Global Explanation Methods}
  \textbf{1. Feature Importance:}
  \begin{itemize}
    \item Ranking features by contribution to predictions
    \item Methods: Permutation importance, SHAP values (aggregated), drop-column
  \end{itemize}

  \vspace{0.5em}

  \textbf{2. Partial Dependence Plots (PDP):}
  \begin{itemize}
    \item Show marginal effect of features on predictions
    \item $\text{PDP}(x_s) = \E_{x_c}[\hat{f}(x_s, x_c)]$
    \item Averaged over all other features
  \end{itemize}

  \vspace{0.5em}

  \textbf{3. Accumulated Local Effects (ALE):}
  \begin{itemize}
    \item Like PDP but handles correlated features better
    \item Based on conditional distributions
  \end{itemize}

  \vspace{0.5em}

  \textbf{4. Model Distillation:}
  \begin{itemize}
    \item Train simpler model to mimic complex model
    \item Interpretable surrogate (decision tree, linear model)
  \end{itemize}
\end{frame}

\begin{frame}{Local Explanation Methods}
  \textbf{1. Individual Conditional Expectation (ICE):}
  \begin{itemize}
    \item Like PDP but for individual instances
    \item Shows heterogeneous effects
  \end{itemize}

  \vspace{0.5em}

  \textbf{2. LIME (Local Interpretable Model-agnostic Explanations):}
  \begin{itemize}
    \item Fit simple model locally around instance
    \item Perturb input and observe model response
  \end{itemize}

  \vspace{0.5em}

  \textbf{3. SHAP (SHapley Additive exPlanations):}
  \begin{itemize}
    \item Game-theoretic approach
    \item Distributes prediction among features fairly
  \end{itemize}

  \vspace{0.5em}

  \textbf{4. Counterfactual Explanations:}
  \begin{itemize}
    \item "What would need to change for different prediction?"
    \item Actionable insights
  \end{itemize}
\end{frame}

% ================================================================
\section{Model-Agnostic Methods}
% ================================================================

\begin{frame}{Permutation Feature Importance}
  \begin{definitionbox}{Permutation Importance}
    Measure importance by randomly permuting a feature and observing the change in model performance.

    \begin{equation}
      \text{FI}_j = \text{Error}(\text{permuted}(X_j)) - \text{Error}(X)
    \end{equation}
  \end{definitionbox}

  \vspace{0.5em}

  \textbf{Algorithm:}
  \begin{enumerate}
    \item Train model on original data
    \item For each feature $j$:
    \begin{enumerate}
      \item Randomly permute values of feature $j$
      \item Compute prediction error
      \item Calculate difference from baseline error
    \end{enumerate}
    \item Rank features by importance
  \end{enumerate}

  \vspace{0.5em}

  \textbf{Advantages:}
  \begin{itemize}
    \item Model-agnostic
    \item Easy to implement
    \item Captures non-linear effects
  \end{itemize}
\end{frame}

\begin{frame}{LIME: Local Interpretable Model-agnostic Explanations}
  \textbf{Core idea:} Approximate complex model locally with interpretable model.

  \vspace{0.5em}

  \begin{definitionbox}{LIME Objective}
    \begin{equation}
      \xi(x) = \argmin_{g \in \mathcal{G}} \mathcal{L}(f, g, \pi_x) + \Omega(g)
    \end{equation}

    where:
    \begin{itemize}
      \item $f$: Black-box model
      \item $g$: Interpretable model (e.g., linear)
      \item $\mathcal{L}$: Measure of how well $g$ approximates $f$
      \item $\pi_x$: Locality kernel (weights nearby samples)
      \item $\Omega(g)$: Complexity penalty
    \end{itemize}
  \end{definitionbox}
\end{frame}

\begin{frame}{LIME Algorithm}
  \textbf{For tabular data:}

  \begin{enumerate}
    \item \textbf{Sample:} Generate perturbed samples around instance $x$
    \begin{itemize}
      \item Create synthetic data by perturbing features
      \item Weight samples by proximity to $x$
    \end{itemize}

    \item \textbf{Predict:} Get black-box predictions for samples

    \item \textbf{Fit:} Train interpretable model (linear regression) on weighted samples

    \item \textbf{Explain:} Use coefficients as feature importance
  \end{enumerate}

  \vspace{0.5em}

  \begin{examplebox}{Example Output}
    "For this loan application (approved):
    \begin{itemize}
      \item Income (\$65K): +0.35 (positive contribution)
      \item Credit score (720): +0.28
      \item Debt ratio (0.25): -0.12
    \end{itemize}"
  \end{examplebox}
\end{frame}

\begin{frame}{SHAP: SHapley Additive exPlanations}
  \textbf{Based on Shapley values from cooperative game theory.}

  \vspace{0.5em}

  \begin{definitionbox}{Shapley Value}
    Contribution of feature $i$ to prediction:
    \begin{equation}
      \phi_i = \sum_{S \subseteq N \setminus \{i\}} \frac{|S|!(|N|-|S|-1)!}{|N|!} [f(S \cup \{i\}) - f(S)]
    \end{equation}

    where $N$ is set of all features, $S$ is a subset not containing $i$.
  \end{definitionbox}

  \vspace{0.5em}

  \textbf{Properties:}
  \begin{itemize}
    \item \textbf{Efficiency:} $\sum_i \phi_i = f(x) - f(\emptyset)$ (prediction explained)
    \item \textbf{Symmetry:} Equal features get equal values
    \item \textbf{Dummy:} Irrelevant features get zero value
    \item \textbf{Additivity:} Values sum correctly
  \end{itemize}
\end{frame}

\begin{frame}{SHAP in Practice}
  \textbf{Computing exact Shapley values is exponential!}

  \vspace{0.5em}

  \textbf{Efficient approximations:}

  \begin{itemize}
    \item \textbf{TreeSHAP:} For tree-based models (RF, XGBoost, LightGBM)
    \begin{itemize}
      \item Polynomial time algorithm
      \item Exact Shapley values
    \end{itemize}

    \item \textbf{KernelSHAP:} Model-agnostic approximation
    \begin{itemize}
      \item Weighted linear regression approach
      \item Similar to LIME but theoretically grounded
    \end{itemize}

    \item \textbf{DeepSHAP:} For neural networks
    \begin{itemize}
      \item Combines DeepLIFT with Shapley values
      \item Fast approximation
    \end{itemize}
  \end{itemize}

  \vspace{0.5em}

  \textbf{Visualization:}
  \begin{itemize}
    \item Waterfall plots (individual predictions)
    \item Summary plots (feature importance)
    \item Dependence plots (feature interactions)
  \end{itemize}
\end{frame}

\begin{frame}{LIME vs. SHAP}
  \begin{center}
  \begin{tabular}{lcc}
    \toprule
    \textbf{Property} & \textbf{LIME} & \textbf{SHAP} \\
    \midrule
    Theoretical foundation & Heuristic & Game theory \\
    Consistency & No guarantee & Guaranteed \\
    Local accuracy & High & High \\
    Computational cost & Low & Medium-High \\
    Additivity & Not guaranteed & Guaranteed \\
    Model-agnostic & Yes & Yes (KernelSHAP) \\
    Specialized versions & Image, text & Tree, Deep \\
    \bottomrule
  \end{tabular}
  \end{center}

  \vspace{1em}

  \begin{block}{When to Use?}
    \begin{itemize}
      \item \textbf{LIME:} Quick explanations, simple use case
      \item \textbf{SHAP:} Rigorous analysis, better properties, worth the computation
    \end{itemize}
  \end{block}
\end{frame}

% ================================================================
\section{Model-Specific Methods}
% ================================================================

\begin{frame}{Intrinsically Interpretable Models}
  \textbf{1. Linear Models:}
  \begin{itemize}
    \item Coefficients directly interpretable
    \item $\hat{y} = \beta_0 + \sum_{j=1}^p \beta_j x_j$
    \item Each $\beta_j$ shows effect of one-unit change in $x_j$
  \end{itemize}

  \vspace{0.5em}

  \textbf{2. Decision Trees:}
  \begin{itemize}
    \item Sequence of if-then rules
    \item Easy to visualize and explain
    \item But: unstable, high variance
  \end{itemize}

  \vspace{0.5em}

  \textbf{3. Generalized Additive Models (GAM):}
  \begin{equation}
    g(\E[y]) = \beta_0 + \sum_{j=1}^p f_j(x_j)
  \end{equation}
  \begin{itemize}
    \item Flexible non-linear effects
    \item Each $f_j$ can be plotted separately
    \item Maintains additivity
  \end{itemize}
\end{frame}

\begin{frame}{Neural Network Interpretation}
  \textbf{1. Gradient-based Methods:}

  \begin{itemize}
    \item \textbf{Saliency Maps:} $\frac{\partial f(x)}{\partial x_i}$
    \begin{itemize}
      \item Shows which inputs affect output most
      \item Commonly used for images
    \end{itemize}

    \item \textbf{Integrated Gradients:}
    \begin{equation}
      \text{IG}_i(x) = (x_i - x_i') \int_{\alpha=0}^1 \frac{\partial f(x' + \alpha(x - x'))}{\partial x_i} d\alpha
    \end{equation}
    \begin{itemize}
      \item Accumulates gradients along path
      \item Satisfies axioms (completeness, sensitivity)
    \end{itemize}

    \item \textbf{Grad-CAM:} For CNNs
    \begin{itemize}
      \item Weighted combination of activation maps
      \item Highlights important regions in images
    \end{itemize}
  \end{itemize}
\end{frame}

\begin{frame}{Attention Mechanisms}
  \textbf{Built-in interpretability in Transformers!}

  \vspace{0.5em}

  \begin{definitionbox}{Attention Weights}
    \begin{equation}
      \text{Attention}(Q, K, V) = \text{softmax}\left(\frac{QK^T}{\sqrt{d_k}}\right)V
    \end{equation}

    Attention weights $\alpha_{ij}$ show how much position $i$ attends to position $j$.
  \end{definitionbox}

  \vspace{0.5em}

  \textbf{Interpretation:}
  \begin{itemize}
    \item Visualize attention matrices
    \item Identify which inputs are important
    \item Understand model reasoning
  \end{itemize}

  \vspace{0.5em}

  \begin{alertbox}{Caveat}
    Attention weights $\neq$ importance!
    \begin{itemize}
      \item Multiple attention heads complicate interpretation
      \item May not reflect actual contribution to prediction
    \end{itemize}
  \end{alertbox}
\end{frame}

% ================================================================
\section{Fairness and Bias Detection}
% ================================================================

\begin{frame}{Algorithmic Fairness}
  \textbf{ML models can perpetuate or amplify bias:}

  \begin{itemize}
    \item Training data reflects historical discrimination
    \item Proxy variables encode protected attributes
    \item Feedback loops reinforce biases
  \end{itemize}

  \vspace{0.5em}

  \begin{examplebox}{Example: COMPAS}
    ProPublica (2016) found COMPAS recidivism algorithm:
    \begin{itemize}
      \item Higher false positive rate for Black defendants
      \item Lower false positive rate for white defendants
      \item Despite being "race-blind"
    \end{itemize}
  \end{examplebox}

  \vspace{0.5em}

  \begin{alertbox}{Legal and Ethical Implications}
    \begin{itemize}
      \item Fair lending laws
      \item Equal employment opportunity
      \item GDPR algorithmic decision-making
    \end{itemize}
  \end{alertbox}
\end{frame}

\begin{frame}{Fairness Definitions}
  \textbf{Let $A$ be protected attribute (race, gender, etc.)}

  \vspace{0.5em}

  \textbf{1. Statistical Parity:}
  \begin{equation}
    P(\hat{Y} = 1 | A = 0) = P(\hat{Y} = 1 | A = 1)
  \end{equation}
  Equal positive prediction rates across groups.

  \vspace{0.3em}

  \textbf{2. Equal Opportunity:}
  \begin{equation}
    P(\hat{Y} = 1 | Y = 1, A = 0) = P(\hat{Y} = 1 | Y = 1, A = 1)
  \end{equation}
  Equal true positive rates (equal recall).

  \vspace{0.3em}

  \textbf{3. Equalized Odds:}
  \begin{align}
    P(\hat{Y} = 1 | Y = y, A = 0) = P(\hat{Y} = 1 | Y = y, A = 1), \quad y \in \{0,1\}
  \end{align}
  Equal TPR and FPR across groups.

  \vspace{0.3em}

  \textbf{4. Calibration:}
  \begin{equation}
    P(Y = 1 | \hat{P} = p, A = a) = p \quad \forall p, a
  \end{equation}
  Predicted probabilities are accurate within groups.
\end{frame}

\begin{frame}{Impossibility Results}
  \begin{theorembox}{Fairness Impossibility (Chouldechova, 2017)}
    Except in degenerate cases, a model cannot simultaneously satisfy:
    \begin{itemize}
      \item Equalized odds (equal TPR and FPR)
      \item Calibration (accurate probabilities)
      \item Different base rates across groups
    \end{itemize}
  \end{theorembox}

  \vspace{0.5em}

  \textbf{Implication:} Must choose which notion of fairness to prioritize!

  \vspace{0.5em}

  \textbf{Trade-offs:}
  \begin{itemize}
    \item Fairness metrics often conflict
    \item No universally "fair" solution
    \item Context-dependent choices
    \item Need stakeholder input
  \end{itemize}
\end{frame}

\begin{frame}{Detecting and Mitigating Bias}
  \textbf{Detection:}
  \begin{enumerate}
    \item Compute fairness metrics for each group
    \item Check for disparate impact
    \item Analyze feature importance by group
    \item Use interpretability tools (SHAP, LIME)
  \end{enumerate}

  \vspace{0.5em}

  \textbf{Mitigation strategies:}

  \begin{itemize}
    \item \textbf{Pre-processing:}
    \begin{itemize}
      \item Re-weight training data
      \item Remove bias from features
    \end{itemize}

    \item \textbf{In-processing:}
    \begin{itemize}
      \item Add fairness constraints to objective
      \item Adversarial debiasing
    \end{itemize}

    \item \textbf{Post-processing:}
    \begin{itemize}
      \item Adjust prediction thresholds per group
      \item Calibration techniques
    \end{itemize}
  \end{itemize}
\end{frame}

% ================================================================
\section{Practical Implementation}
% ================================================================

\begin{frame}[fragile]{Python Tools for XAI}
  \textbf{Popular libraries:}

  \vspace{0.5em}

  \textbf{1. SHAP:}
  \begin{lstlisting}[language=Python]
import shap

# For tree models
explainer = shap.TreeExplainer(model)
shap_values = explainer.shap_values(X_test)

# Visualizations
shap.summary_plot(shap_values, X_test)
shap.waterfall_plot(shap_values[0])
shap.dependence_plot("feature_name", shap_values, X_test)
  \end{lstlisting}

  \vspace{0.5em}

  \textbf{2. LIME:}
  \begin{lstlisting}[language=Python]
from lime.lime_tabular import LimeTabularExplainer

explainer = LimeTabularExplainer(X_train, mode='classification')
explanation = explainer.explain_instance(X_test[0],
                                         model.predict_proba)
explanation.show_in_notebook()
  \end{lstlisting}
\end{frame}

\begin{frame}[fragile]{More Tools}
  \textbf{3. InterpretML (Microsoft):}
  \begin{lstlisting}[language=Python]
from interpret.glassbox import ExplainableBoostingClassifier
from interpret import show

# Train interpretable model
ebm = ExplainableBoostingClassifier()
ebm.fit(X_train, y_train)

# Global explanation
ebm_global = ebm.explain_global()
show(ebm_global)

# Local explanation
ebm_local = ebm.explain_local(X_test, y_test)
show(ebm_local)
  \end{lstlisting}

  \vspace{0.5em}

  \textbf{4. Fairlearn (Microsoft):}
  \begin{lstlisting}[language=Python]
from fairlearn.metrics import MetricFrame

# Compute metrics by group
metric_frame = MetricFrame(
    metrics={'accuracy': accuracy_score, 'fpr': false_positive_rate},
    y_true=y_test,
    y_pred=y_pred,
    sensitive_features=sensitive_features
)
print(metric_frame.by_group)
  \end{lstlisting}
\end{frame}

\begin{frame}{Best Practices}
  \textbf{1. Choose the right explanation method:}
  \begin{itemize}
    \item Model type (tree-based, neural network, etc.)
    \item Audience (technical vs. non-technical)
    \item Purpose (debugging, transparency, fairness)
  \end{itemize}

  \vspace{0.5em}

  \textbf{2. Validate explanations:}
  \begin{itemize}
    \item Check consistency across methods
    \item Test with synthetic data
    \item Compare to domain knowledge
  \end{itemize}

  \vspace{0.5em}

  \textbf{3. Communicate effectively:}
  \begin{itemize}
    \item Use visualizations
    \item Provide context
    \item Avoid over-interpretation
  \end{itemize}

  \vspace{0.5em}

  \textbf{4. Consider computational cost:}
  \begin{itemize}
    \item SHAP can be expensive for large datasets
    \item Use approximations when needed
    \item Cache results when possible
  \end{itemize}
\end{frame}

% ================================================================
\section{Conclusion}
% ================================================================

\begin{frame}{Summary}
  \textbf{Key Takeaways:}

  \begin{itemize}
    \item Interpretability is crucial for trust, debugging, and fairness
    \item \textbf{Global explanations:} Understand model behavior overall
    \item \textbf{Local explanations:} Explain individual predictions
    \item \textbf{SHAP:} Theoretically sound, widely applicable
    \item \textbf{LIME:} Fast, intuitive, model-agnostic
    \item \textbf{Fairness:} Multiple definitions, inherent trade-offs
  \end{itemize}

  \vspace{0.5em}

  \begin{alertbox}{The Future of XAI}
    \begin{itemize}
      \item Regulatory requirements increasing
      \item Better integration into ML pipelines
      \item Standardization of methods and metrics
      \item Causal explanations beyond correlations
    \end{itemize}
  \end{alertbox}
\end{frame}

\begin{frame}{Further Reading}
  \textbf{Books:}
  \begin{itemize}
    \item Molnar (2022). \textit{Interpretable Machine Learning}
    \item Barocas, Hardt, Narayanan (2019). \textit{Fairness and Machine Learning}
  \end{itemize}

  \vspace{0.5em}

  \textbf{Key Papers:}
  \begin{itemize}
    \item Ribeiro et al. (2016). "Why Should I Trust You? Explaining Predictions" (LIME)
    \item Lundberg \& Lee (2017). "A Unified Approach to Interpreting Model Predictions" (SHAP)
    \item Doshi-Velez \& Kim (2017). "Towards A Rigorous Science of Interpretable ML"
    \item Chouldechova (2017). "Fair Prediction with Disparate Impact"
  \end{itemize}

  \vspace{0.5em}

  \textbf{Tools:}
  \begin{itemize}
    \item SHAP: \url{https://github.com/slundberg/shap}
    \item LIME: \url{https://github.com/marcotcr/lime}
    \item InterpretML: \url{https://interpret.ml/}
    \item Fairlearn: \url{https://fairlearn.org/}
  \end{itemize}
\end{frame}

\acknowledgmentsslide{
  \item ESMAD for institutional support
  \item Mysense.ai for applied AI ethics work
  \item XAI research community
}

\contactslide

\end{document}
