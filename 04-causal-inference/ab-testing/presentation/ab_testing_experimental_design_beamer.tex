% ================================================================
% A/B TESTING & EXPERIMENTAL DESIGN
% Comprehensive presentation on experimentation and causal inference
%
% Topics Covered:
% 1. Experimental Design Fundamentals
% 2. Classical Hypothesis Testing Approach
% 3. Sample Size and Statistical Power
% 4. Bayesian A/B Testing
% 5. Sequential Testing and Early Stopping
% 6. Multi-Armed Bandits
% 7. Variance Reduction Techniques
% 8. Multiple Testing and Corrections
% 9. Advanced Topics and Case Studies
%
% Total: ~50 slides
% Author: Diogo Ribeiro
% Date: January 2025
% ================================================================

\section{A/B Testing \& Experimental Design}

% ================================================================
% PART 1: FUNDAMENTALS
% ================================================================

\begin{frame}{What is A/B Testing?}
\textbf{The gold standard for causal inference}

\begin{columns}
\begin{column}{0.5\textwidth}
\textbf{Definition:}

A \textbf{randomized controlled experiment} comparing two (or more) versions to determine which performs better.

\vspace{0.2cm}

\textbf{The Classic Setup:}
\begin{itemize}
\item \textcolor{forest}{\textbf{Control (A):}} Current version (baseline)
\item \textcolor{crimson}{\textbf{Treatment (B):}} New version (variant)
\item \textbf{Random assignment:} Users randomly assigned to A or B
\item \textbf{Measure outcome:} Compare conversion, revenue, engagement, etc.
\end{itemize}

\vspace{0.3cm}

\textbf{Why Randomization?}

\textbf{Eliminates selection bias!}

\[
\E[Y^1 | T=1] - \E[Y^0 | T=0] = \text{ATE}
\]

Only with randomization:
\[
\E[Y^1 | T=1] = \E[Y^1] \quad \text{and} \quad \E[Y^0 | T=0] = \E[Y^0]
\]

$\to$ Can estimate true causal effect.
\end{column}

\begin{column}{0.5\textwidth}
\textbf{Example: Website Button Color}

\begin{figure}
\centering
\begin{tikzpicture}[scale=0.7]
% Population
\node[draw, ellipse, fill=navyblue!20, minimum width=4cm, minimum height=1.5cm] (pop) at (3,5) {All Users};

% Random assignment
\draw[->, very thick] (1.5,4.5) -- (1.5,3.5) node[midway, left] {\tiny Random};
\draw[->, very thick] (4.5,4.5) -- (4.5,3.5) node[midway, right] {\tiny Random};

% Groups
\node[draw, rectangle, fill=forest!20, minimum width=2cm, minimum height=1.2cm] (a) at (1.5,2.5) {Control (A)\\Blue button\\n=1000};
\node[draw, rectangle, fill=crimson!20, minimum width=2cm, minimum height=1.2cm] (b) at (4.5,2.5) {Treatment (B)\\Red button\\n=1000};

% Outcomes
\node[draw, circle, fill=gold!20] (oa) at (1.5,1) {CVR: 5.2\%};
\node[draw, circle, fill=gold!20] (ob) at (4.5,1) {CVR: 5.8\%};

\draw[->, thick] (a) -- (oa);
\draw[->, thick] (b) -- (ob);

% Comparison
\draw[<->, very thick, purple] (oa) -- node[below] {\tiny Compare} (ob);
\end{tikzpicture}
\end{figure}

\textbf{Result:} Red button increases conversion by 0.6 percentage points (5.8\% - 5.2\%)

\vspace{0.2cm}

\textbf{Key Question:}

Is this difference real or due to chance?

$\to$ Statistical hypothesis testing!
\end{column}
\end{columns}

\vspace{0.2cm}

\begin{alertblock}{Causation, Not Just Correlation}
Randomization ensures treatment \textbf{causes} the difference in outcomes!
\end{alertblock}
\end{frame}

\begin{frame}{Why A/B Testing Matters}
\textbf{From intuition to data-driven decisions}

\begin{columns}
\begin{column}{0.5\textwidth}
\textbf{The Problem with "Just Ship It":}

\begin{itemize}
\item[$\times$] Opinions vary (HiPPO: Highest Paid Person's Opinion)
\item[$\times$] Intuition is often wrong
\item[$\times$] Can't measure causal impact
\item[$\times$] Risk of harming metrics
\end{itemize}

\vspace{0.3cm}

\textbf{Famous A/B Testing Successes:}

\begin{enumerate}
\item \textbf{Google: 41 Shades of Blue}
   \begin{itemize}
   \item Tested 41 shades for ad links
   \item Found optimal shade
   \item \$200M additional revenue/year
   \end{itemize}

\item \textbf{Netflix: Artwork Testing}
   \begin{itemize}
   \item Different thumbnails for shows
   \item 30\% increase in viewing
   \end{itemize}

\item \textbf{Amazon: Free Shipping}
   \begin{itemize}
   \item Tested \$25 threshold
   \item Massive increase in orders
   \end{itemize}

\item \textbf{Bing: Slow Page Load}
   \begin{itemize}
   \item Tested 200ms delay
   \item 4.3\% drop in revenue
   \item Showed cost of latency
   \end{itemize}
\end{enumerate}
\end{column}

\begin{column}{0.5\textwidth}
\textbf{When to Use A/B Testing:}

\begin{table}
\tiny
\begin{tabular}{|l|p{3.5cm}|}
\hline
\rowcolor{navyblue!20}
\textbf{Good For} & \textbf{Example} \\
\hline
Product changes & New checkout flow \\
\hline
UI/UX variations & Button color, layout \\
\hline
Pricing experiments & \$9.99 vs. \$10.99 \\
\hline
Algorithm changes & Recommendation system \\
\hline
Marketing campaigns & Email subject lines \\
\hline
Feature launches & New social feature \\
\hline
\end{tabular}
\end{table}

\vspace{0.3cm}

\textbf{When NOT to Use A/B Testing:}

\begin{itemize}
\item[$\times$] Not enough traffic (insufficient power)
\item[$\times$] Long-term effects (network effects)
\item[$\times$] Ethical concerns (medical trials need IRB)
\item[$\times$] Technical impossibility (can't randomize)
\item[$\times$] Obvious improvements (remove broken feature)
\end{itemize}

\vspace{0.2cm}

\textbf{Requirements for Valid A/B Test:}

\begin{enumerate}
\item Random assignment
\item Sufficient sample size
\item No contamination between groups
\item Stable treatment effect
\item Measurable outcome metric
\end{enumerate}
\end{column}
\end{columns}
\end{frame}

% ================================================================
% PART 2: CLASSICAL HYPOTHESIS TESTING
% ================================================================

\begin{frame}{The Hypothesis Testing Framework}
\textbf{Frequentist approach to A/B testing}

\begin{columns}
\begin{column}{0.5\textwidth}
\textbf{Setup:}

\begin{enumerate}
\item \textbf{Null Hypothesis ($H_0$):}
   \[
   H_0: \mu_B - \mu_A = 0
   \]
   No difference between variants

\item \textbf{Alternative Hypothesis ($H_1$):}
   \[
   H_1: \mu_B - \mu_A \neq 0
   \]
   There is a difference

\item \textbf{Significance Level ($\alpha$):}
   \begin{itemize}
   \item Probability of Type I error (false positive)
   \item Typically $\alpha = 0.05$ (5\%)
   \end{itemize}

\item \textbf{Test Statistic:}
   \[
   z = \frac{\hat{p}_B - \hat{p}_A}{\sqrt{\hat{p}(1-\hat{p})\left(\frac{1}{n_A} + \frac{1}{n_B}\right)}}
   \]
   for proportions (e.g., conversion rate)

\item \textbf{Decision Rule:}
   \begin{itemize}
   \item If $p$-value $< \alpha$: Reject $H_0$
   \item Otherwise: Fail to reject $H_0$
   \end{itemize}
\end{enumerate}
\end{column}

\begin{column}{0.5\textwidth}
\textbf{Two Types of Errors:}

\begin{table}
\tiny
\begin{tabular}{|l|c|c|}
\hline
\rowcolor{navyblue!20}
 & \textbf{$H_0$ True} & \textbf{$H_0$ False} \\
\hline
\textbf{Reject $H_0$} & \textcolor{crimson}{Type I Error} ($\alpha$) & Correct (Power) \\
\hline
\textbf{Fail to reject} & Correct & \textcolor{crimson}{Type II Error} ($\beta$) \\
\hline
\end{tabular}
\end{table}

\begin{itemize}
\item \textbf{Type I Error ($\alpha$):} False positive - say there's an effect when there isn't
\item \textbf{Type II Error ($\beta$):} False negative - miss a real effect
\item \textbf{Statistical Power:} $1 - \beta$ - probability of detecting true effect
\end{itemize}

\vspace{0.3cm}

\textbf{Example Calculation:}

\begin{verbatim}
from scipy import stats

# Control: 520 conversions / 10000 users
# Treatment: 580 conversions / 10000 users

n_A, n_B = 10000, 10000
conv_A, conv_B = 520, 580
p_A, p_B = conv_A/n_A, conv_B/n_B

# Pooled proportion
p_pool = (conv_A + conv_B) / (n_A + n_B)

# Standard error
se = np.sqrt(p_pool * (1-p_pool) *
             (1/n_A + 1/n_B))

# Z-statistic
z = (p_B - p_A) / se

# P-value (two-tailed)
p_value = 2 * (1 - stats.norm.cdf(abs(z)))

print(f"p-value: {p_value:.4f}")
# p-value: 0.0156 < 0.05 → Significant!
\end{verbatim}
\end{column}
\end{columns}
\end{frame}

\begin{frame}{Sample Size and Statistical Power}
\textbf{How many users do I need?}

\begin{columns}
\begin{column}{0.5\textwidth}
\textbf{Power Analysis:}

Determines sample size needed to detect effect with high probability.

\textbf{Inputs:}
\begin{enumerate}
\item $\alpha$: Significance level (0.05)
\item $1-\beta$: Desired power (0.80 typical)
\item $\delta$: Minimum detectable effect (MDE)
\item $\sigma$: Variance (or baseline rate)
\end{enumerate}

\vspace{0.2cm}

\textbf{For Proportions (Conversion Rate):}

\[
n = \frac{(z_{1-\alpha/2} + z_{1-\beta})^2 \cdot 2p(1-p)}{(\delta)^2}
\]

where $p$ is baseline conversion rate.

\vspace{0.2cm}

\textbf{Key Insights:}
\begin{itemize}
\item Smaller $\delta$ (effect) $\to$ larger $n$ needed
\item Higher power $\to$ larger $n$ needed
\item Lower baseline rate $\to$ larger $n$ needed
\end{itemize}

\vspace{0.2cm}

\textbf{Example:}

Baseline CVR = 5\%, want to detect 10\% relative lift (0.5pp absolute):
\[
n \approx 15,736 \text{ per group}
\]
\end{column}

\begin{column}{0.5\textwidth}
\textbf{Power Calculation (Python):}

\begin{verbatim}
from statsmodels.stats.power import
    zt_ind_solve_power

# Parameters
alpha = 0.05
power = 0.80
baseline_rate = 0.05
relative_lift = 0.10  # 10% improvement

# Effect size (Cohen's h)
p1 = baseline_rate
p2 = baseline_rate * (1 + relative_lift)
effect_size = 2 * (np.arcsin(np.sqrt(p2))
                   - np.arcsin(np.sqrt(p1)))

# Calculate sample size
n_per_group = zt_ind_solve_power(
    effect_size=effect_size,
    alpha=alpha,
    power=power,
    alternative='two-sided'
)

print(f"Need {n_per_group:.0f} users per group")
# Need 15,736 users per group
\end{verbatim}

\vspace{0.2cm}

\textbf{Practical Considerations:}

\begin{itemize}
\item \textbf{Traffic:} Do we have enough users?
\item \textbf{Duration:} How long to reach sample size?
\item \textbf{Seasonality:} Run full week cycles
\item \textbf{Attrition:} Account for dropout
\end{itemize}
\end{column}
\end{columns}

\vspace{0.2cm}

\begin{alertblock}{Rule of Thumb}
Smaller effects require exponentially more data. Be realistic about MDE!
\end{alertblock}
\end{frame}

% ================================================================
% PART 3: BAYESIAN A/B TESTING
% ================================================================

\begin{frame}{Bayesian A/B Testing}
\textbf{The modern alternative to p-values}

\begin{columns}
\begin{column}{0.5\textwidth}
\textbf{Frequentist vs. Bayesian:}

\begin{table}
\tiny
\begin{tabular}{|p{2.3cm}|p{2.3cm}|}
\hline
\rowcolor{navyblue!20}
\textbf{Frequentist} & \textbf{Bayesian} \\
\hline
P(data | $H_0$ true) & P($H_1$ true | data) \\
\hline
p-value & Probability B beats A \\
\hline
Reject or not reject & Posterior distribution \\
\hline
Fixed sample size & Can stop anytime \\
\hline
Hard to interpret & Intuitive \\
\hline
\end{tabular}
\end{table}

\vspace{0.3cm}

\textbf{Bayesian Approach:}

\begin{enumerate}
\item \textbf{Prior belief:}
   \[
   p_A \sim \text{Beta}(\alpha_A, \beta_A)
   \]
   \[
   p_B \sim \text{Beta}(\alpha_B, \beta_B)
   \]

\item \textbf{Observe data:}
   \begin{itemize}
   \item Control: $x_A$ successes, $n_A$ trials
   \item Treatment: $x_B$ successes, $n_B$ trials
   \end{itemize}

\item \textbf{Update to posterior:}
   \[
   p_A | \text{data} \sim \text{Beta}(\alpha_A + x_A, \beta_A + n_A - x_A)
   \]
   \[
   p_B | \text{data} \sim \text{Beta}(\alpha_B + x_B, \beta_B + n_B - x_B)
   \]
\end{enumerate}
\end{column}

\begin{column}{0.5\textwidth}
\begin{enumerate}
\setcounter{enumi}{3}
\item \textbf{Compute probability B beats A:}
   \[
   P(p_B > p_A | \text{data})
   \]
   via Monte Carlo simulation
\end{enumerate}

\vspace{0.2cm}

\textbf{Implementation:}

\begin{verbatim}
import numpy as np
from scipy import stats

# Data
n_A, x_A = 10000, 520
n_B, x_B = 10000, 580

# Priors (uninformative)
alpha_prior, beta_prior = 1, 1

# Posteriors
posterior_A = stats.beta(
    alpha_prior + x_A,
    beta_prior + n_A - x_A
)
posterior_B = stats.beta(
    alpha_prior + x_B,
    beta_prior + n_B - x_B
)

# Monte Carlo
samples_A = posterior_A.rvs(100000)
samples_B = posterior_B.rvs(100000)

# Probability B > A
prob_B_beats_A = (samples_B > samples_A).mean()
print(f"P(B > A) = {prob_B_beats_A:.3f}")
# P(B > A) = 0.984

# Expected loss if we choose A
expected_loss_A = (samples_B - samples_A)[
    samples_B > samples_A].mean()
print(f"Expected loss: {expected_loss_A:.4f}")
# Expected loss: 0.0063
\end{verbatim}
\end{column}
\end{columns}
\end{frame}

\begin{frame}{Bayesian Decision Making}
\textbf{Beyond just probability - expected value}

\begin{columns}
\begin{column}{0.5\textwidth}
\textbf{Decision Criteria:}

\begin{enumerate}
\item \textcolor{forest}{\textbf{Probability to Beat:}}
   \[
   P(p_B > p_A) > 0.95
   \]
   95\% confident B is better

\item \textcolor{navyblue}{\textbf{Expected Loss:}}
   \[
   \E[\max(0, p_A - p_B)] < \epsilon
   \]
   If we're wrong, loss is small

\item \textcolor{crimson}{\textbf{Credible Interval:}}
   \[
   P(p_B - p_A \in [L, U]) = 0.95
   \]
   95\% interval for difference
\end{enumerate}

\vspace{0.3cm}

\textbf{Visualizing Posteriors:}

\begin{figure}
\centering
\begin{tikzpicture}[scale=0.7]
\draw[->] (0,0) -- (6,0) node[right] {\tiny CVR};
\draw[->] (0,0) -- (0,3) node[above] {\tiny Density};

% Posterior A
\draw[thick, forest, fill=forest!20, opacity=0.5] plot[smooth, domain=1.5:4.5]
  (\x, {2.5*exp(-3*(\x-2.8)^2)});
\node at (2.8,2.8) {\tiny \textcolor{forest}{A}};

% Posterior B
\draw[thick, crimson, fill=crimson!20, opacity=0.5] plot[smooth, domain=2:5]
  (\x, {2.5*exp(-3*(\x-3.4)^2)});
\node at (3.4,2.8) {\tiny \textcolor{crimson}{B}};

% Overlap region
\draw[dashed] (3.1,0) -- (3.1,2);
\node at (3.1,-0.3) {\tiny Overlap};
\end{tikzpicture}
\end{figure}
\end{column}

\begin{column}{0.5\textwidth}
\textbf{Full Bayesian Analysis:}

\begin{verbatim}
# Credible interval for difference
diff_samples = samples_B - samples_A
ci_95 = np.percentile(diff_samples,
                      [2.5, 97.5])
print(f"95% CI for lift: {ci_95}")
# 95% CI: [0.0018, 0.0102]

# Probability of at least 10% relative lift
relative_lift = (samples_B - samples_A) /
                samples_A
prob_10pct_lift = (relative_lift > 0.10).mean()
print(f"P(lift > 10%) = {prob_10pct_lift:.3f}")

# Expected improvement
expected_lift = diff_samples.mean()
print(f"Expected lift: {expected_lift:.4f}")

# Risk analysis
prob_B_worse = (samples_B < samples_A).mean()
expected_loss_B = (samples_A - samples_B)[
    samples_A > samples_B].mean()
print(f"Risk of B: {prob_B_worse:.3f}")
print(f"Expected loss if wrong:
      {expected_loss_B:.4f}")
\end{verbatim}

\vspace{0.2cm}

\textbf{Advantages:}
\begin{itemize}
\item[$\checkmark$] Intuitive interpretation
\item[$\checkmark$] Can incorporate prior knowledge
\item[$\checkmark$] Flexible stopping rules
\item[$\checkmark$] Quantifies uncertainty
\item[$\checkmark$] Can compute expected value
\end{itemize}
\end{column}
\end{columns}
\end{frame}

% ================================================================
% PART 4: SEQUENTIAL TESTING
% ================================================================

\begin{frame}{Sequential Testing and Early Stopping}
\textbf{Can we stop the experiment early?}

\begin{columns}
\begin{column}{0.5\textwidth}
\textbf{The Peeking Problem:}

\textcolor{crimson}{\textbf{DON'T:}} Check p-value every day and stop when $p < 0.05$

\textbf{Why not?}
\begin{itemize}
\item Inflates Type I error rate!
\item With enough peeks, random noise looks significant
\item Actual $\alpha$ can be 20\%+ instead of 5\%
\end{itemize}

\vspace{0.3cm}

\textbf{Sequential Probability Ratio Test (SPRT):}

Allows continuous monitoring with controlled error rates.

\textbf{Test statistic:}
\[
\Lambda_t = \frac{P(X_1, \ldots, X_t | H_1)}{P(X_1, \ldots, X_t | H_0)}
\]

\textbf{Decision rules:}
\begin{itemize}
\item If $\Lambda_t \geq B$: Reject $H_0$ (B wins)
\item If $\Lambda_t \leq A$: Accept $H_0$ (no difference)
\item Otherwise: Continue testing
\end{itemize}

where $A = \frac{\beta}{1-\alpha}$ and $B = \frac{1-\beta}{\alpha}$
\end{column}

\begin{column}{0.5\textwidth}
\textbf{Always Valid P-Values:}

Adjusted p-values that remain valid despite continuous monitoring.

\textbf{Methods:}
\begin{enumerate}
\item \textbf{Bonferroni Correction:}
   \begin{itemize}
   \item Divide $\alpha$ by number of looks
   \item Conservative
   \end{itemize}

\item \textbf{Alpha Spending Function:}
   \begin{itemize}
   \item Allocate significance level over time
   \item More powerful than Bonferroni
   \end{itemize}

\item \textbf{Bayesian Approach:}
   \begin{itemize}
   \item No peeking problem!
   \item Can check anytime
   \end{itemize}
\end{enumerate}

\vspace{0.2cm}

\textbf{Practical Guidelines:}

\begin{itemize}
\item \textbf{Pre-specify:} When you'll check
\item \textbf{Limit looks:} e.g., 25\%, 50\%, 75\%, 100\%
\item \textbf{Use adjusted thresholds:} Lower $\alpha$ for early looks
\item \textbf{Document:} All interim analyses
\end{itemize}

\vspace{0.2cm}

\textbf{When to Stop Early:}

\begin{itemize}
\item \textcolor{forest}{\textbf{For efficacy:}} Clear winner emerges
\item \textcolor{crimson}{\textbf{For futility:}} Unlikely to find effect
\item \textcolor{purple}{\textbf{For harm:}} Treatment causing damage
\end{itemize}
\end{column}
\end{columns}

\vspace{0.2cm}

\begin{alertblock}{Warning}
Continuous peeking without correction is statistical malpractice!
\end{alertblock}
\end{frame}

% ================================================================
% PART 5: MULTI-ARMED BANDITS
% ================================================================

\begin{frame}{Multi-Armed Bandits}
\textbf{Exploration vs. exploitation in online experiments}

\begin{columns}
\begin{column}{0.5\textwidth}
\textbf{The Bandit Problem:}

You have $K$ slot machines (arms). Each pull gives random reward. Goal: Maximize total reward.

\textbf{Trade-off:}
\begin{itemize}
\item \textbf{Explore:} Try arms to learn which is best
\item \textbf{Exploit:} Use best arm found so far
\end{itemize}

\vspace{0.2cm}

\textbf{Why for A/B Testing?}

\begin{itemize}
\item Traditional A/B test: 50/50 split entire time
\item Bandit: Adapt allocation to favor winning variant
\item \textcolor{forest}{\textbf{Minimize regret!}}
\end{itemize}

\vspace{0.2cm}

\textbf{Regret:}
\[
R_T = T \mu^* - \sum_{t=1}^T \mu_{a_t}
\]
where $\mu^*$ is best arm's mean.

\vspace{0.3cm}

\textbf{Popular Algorithms:}

\begin{enumerate}
\item \textcolor{forest}{\textbf{$\epsilon$-Greedy:}}
   \begin{itemize}
   \item With prob $\epsilon$: random arm
   \item With prob $1-\epsilon$: best arm so far
   \item Simple but effective
   \end{itemize}
\end{enumerate}
\end{column}

\begin{column}{0.5\textwidth}
\begin{enumerate}
\setcounter{enumi}{1}
\item \textcolor{navyblue}{\textbf{UCB (Upper Confidence Bound):}}
   \[
   a_t = \argmax_a \left[\hat{\mu}_a + \sqrt{\frac{2\log t}{n_a}}\right]
   \]
   \begin{itemize}
   \item Optimistic exploration
   \item Theoretically optimal regret
   \end{itemize}

\item \textcolor{crimson}{\textbf{Thompson Sampling:}}
   \begin{itemize}
   \item Bayesian approach
   \item Sample from posterior, pick best
   \item Excellent empirical performance
   \end{itemize}
\end{enumerate}

\vspace{0.2cm}

\textbf{Thompson Sampling Implementation:}

\begin{verbatim}
# Beta-Bernoulli bandit
successes = [1, 1]  # Alpha
failures = [1, 1]   # Beta

for t in range(n_trials):
    # Sample from posteriors
    theta_A = np.random.beta(
        successes[0], failures[0])
    theta_B = np.random.beta(
        successes[1], failures[1])

    # Choose best sample
    arm = 0 if theta_A > theta_B else 1

    # Pull arm, observe reward
    reward = pull_arm(arm)

    # Update posterior
    if reward == 1:
        successes[arm] += 1
    else:
        failures[arm] += 1
\end{verbatim}
\end{column}
\end{columns}
\end{frame}

\begin{frame}{Bandits vs. A/B Tests}
\textbf{When to use each approach}

\begin{columns}
\begin{column}{0.5\textwidth}
\textbf{Comparison:}

\begin{table}
\tiny
\begin{tabular}{|l|p{2cm}|p{2cm}|}
\hline
\rowcolor{navyblue!20}
\textbf{Aspect} & \textbf{A/B Test} & \textbf{Bandit} \\
\hline
Allocation & Fixed (50/50) & Adaptive \\
\hline
Goal & Learn best & Maximize reward \\
\hline
Regret & High & Low \\
\hline
Statistical power & High & Lower \\
\hline
Complexity & Simple & More complex \\
\hline
When to use & Clear winner matters & Ongoing optimization \\
\hline
\end{tabular}
\end{table}

\vspace{0.3cm}

\textbf{Use A/B Test When:}
\begin{itemize}
\item Need definitive answer
\item Making long-term product decision
\item Network effects matter
\item Statistical rigor critical
\item One-time decision
\end{itemize}

\vspace{0.2cm}

\textbf{Use Bandits When:}
\begin{itemize}
\item Minimizing cost of exploration
\item Continuous optimization
\item Many variants to test
\item Fast iteration needed
\item Revenue/cost directly matters
\end{itemize}
\end{column}

\begin{column}{0.5\textwidth}
\textbf{Hybrid Approaches:}

\begin{enumerate}
\item \textbf{A/B test first, then bandit:}
   \begin{itemize}
   \item Run fixed A/B for statistical power
   \item Switch to bandit for exploitation
   \end{itemize}

\item \textbf{Contextual bandits:}
   \begin{itemize}
   \item Personalization
   \item Use user features to predict best arm
   \item E.g., LinUCB, neural bandits
   \end{itemize}
\end{enumerate}

\vspace{0.2cm}

\textbf{Real-World Example: Content Recommendation}

\begin{itemize}
\item \textbf{Scenario:} Recommend articles on homepage
\item \textbf{Challenge:} New articles constantly added
\item \textbf{Solution:} Contextual bandit
  \begin{itemize}
  \item Context: User demographics, history
  \item Action: Which article to show
  \item Reward: Click (1) or no click (0)
  \end{itemize}
\item \textbf{Benefit:} Adapt in real-time, maximize clicks
\end{itemize}

\vspace{0.2cm}

\textbf{Pitfalls:}
\begin{itemize}
\item[$\times$] Harder to reason about statistical significance
\item[$\times$] Can't easily compute confidence intervals
\item[$\times$] May get stuck in local optima
\item[$\times$] Requires more engineering
\end{itemize}
\end{column}
\end{columns}
\end{frame}

% ================================================================
% PART 6: VARIANCE REDUCTION
% ================================================================

\begin{frame}{Variance Reduction Techniques}
\textbf{Improving sensitivity of experiments}

\begin{columns}
\begin{column}{0.5\textwidth}
\textbf{Why Variance Reduction?}

\textbf{Lower variance $\to$ Smaller sample size needed!}

\[
n \propto \frac{\sigma^2}{\delta^2}
\]

Reducing $\sigma$ by 50\% $\to$ Need 75\% fewer samples.

\vspace{0.3cm}

\textbf{1. CUPED (Controlled-experiment Using Pre-Experiment Data):}

\textbf{Idea:} Use pre-treatment covariate to reduce variance.

\textbf{Adjusted metric:}
\[
Y_{adj} = Y - \theta(X - \E[X])
\]
where:
\begin{itemize}
\item $Y$: Post-treatment outcome
\item $X$: Pre-treatment covariate
\item $\theta = \text{Cov}(X,Y) / \text{Var}(X)$
\end{itemize}

\textbf{Example:}
\begin{itemize}
\item $Y$: Revenue during experiment
\item $X$: Revenue before experiment
\item Adjust for baseline spending
\end{itemize}

\textbf{Variance Reduction:}
\[
\text{Var}(Y_{adj}) = \text{Var}(Y)(1 - \rho^2)
\]
where $\rho$ is correlation between $X$ and $Y$.
\end{column}

\begin{column}{0.5\textwidth}
\textbf{CUPED Implementation:}

\begin{verbatim}
# Pre-experiment data
X_control = df_control['pre_metric']
X_treatment = df_treatment['pre_metric']

# Post-experiment data
Y_control = df_control['post_metric']
Y_treatment = df_treatment['post_metric']

# Combine for theta estimation
X_all = np.concatenate([X_control,
                        X_treatment])
Y_all = np.concatenate([Y_control,
                        Y_treatment])

# Estimate theta
theta = np.cov(X_all, Y_all)[0,1] /
        np.var(X_all)

# Adjust outcomes
Y_control_adj = Y_control - theta *
    (X_control - X_all.mean())
Y_treatment_adj = Y_treatment - theta *
    (X_treatment - X_all.mean())

# Test adjusted metrics
t_stat, p_value = stats.ttest_ind(
    Y_treatment_adj, Y_control_adj)

# Variance reduction
var_reduction = 1 - (np.var(Y_control_adj)
    / np.var(Y_control))
print(f"Variance reduced by
      {var_reduction*100:.1f}%")
\end{verbatim}
\end{column}
\end{columns}
\end{frame}

\begin{frame}{More Variance Reduction Techniques}
\textbf{Additional methods for sensitive experiments}

\begin{columns}
\begin{column}{0.5\textwidth}
\textbf{2. Stratification:}

Divide population into strata, balance across groups.

\begin{itemize}
\item \textbf{Why:} Ensures groups are balanced
\item \textbf{How:} Randomize within strata
\item \textbf{Example:} Stratify by country, platform, user segment
\end{itemize}

\textbf{Stratified analysis:}
\[
\hat{\tau} = \sum_{s=1}^S w_s (\bar{Y}_{s,1} - \bar{Y}_{s,0})
\]
where $w_s$ is proportion in stratum $s$.

\vspace{0.3cm}

\textbf{3. Regression Adjustment:}

Control for covariates in analysis:
\[
Y_i = \alpha + \tau T_i + \beta X_i + \epsilon_i
\]

\begin{itemize}
\item $T_i$: Treatment indicator
\item $X_i$: Covariates (demographics, behavior)
\item $\tau$: Treatment effect (more precise!)
\end{itemize}

\vspace{0.2cm}

\textbf{4. Difference-in-Differences:}

For panel data (before-after):
\[
\tau = (\bar{Y}_{T,\text{post}} - \bar{Y}_{T,\text{pre}}) - (\bar{Y}_{C,\text{post}} - \bar{Y}_{C,\text{pre}})
\]
\end{column}

\begin{column}{0.5\textwidth}
\textbf{5. Paired Experiments:}

Match similar units, randomize within pairs.

\textbf{Example: Geo Experiments}
\begin{itemize}
\item Pairs of similar cities
\item One city treatment, one control
\item Reduces geographic variation
\end{itemize}

\vspace{0.2cm}

\textbf{Comparison of Methods:}

\begin{table}
\tiny
\begin{tabular}{|l|c|c|}
\hline
\rowcolor{navyblue!20}
\textbf{Method} & \textbf{Complexity} & \textbf{Variance Reduction} \\
\hline
CUPED & Medium & 20-50\% typical \\
\hline
Stratification & Low & 10-30\% \\
\hline
Regression Adj & Medium & Depends on $R^2$ \\
\hline
DiD & Low & 30-60\% \\
\hline
Pairing & High & 40-70\% \\
\hline
\end{tabular}
\end{table}

\vspace{0.3cm}

\textbf{Best Practices:}

\begin{enumerate}
\item Use multiple methods when possible
\item Pre-specify in analysis plan
\item Check assumptions (parallel trends for DiD)
\item Report both adjusted and unadjusted
\item Document all adjustments
\end{enumerate}

\vspace{0.2cm}

\begin{alertblock}{Key Benefit}
Variance reduction can cut sample size requirements by 50\%+, meaning faster experiments!
\end{alertblock}
\end{column}
\end{columns}
\end{frame}

% ================================================================
% PART 7: MULTIPLE TESTING
% ================================================================

\begin{frame}{Multiple Testing Problem}
\textbf{When you test many hypotheses simultaneously}

\begin{columns}
\begin{column}{0.5\textwidth}
\textbf{The Problem:}

Test 20 metrics at $\alpha = 0.05$:
\[
P(\text{at least 1 false positive}) = 1 - (1-0.05)^{20} \approx 0.64
\]

\textcolor{crimson}{\textbf{64\% chance of false positive!}}

\vspace{0.2cm}

\textbf{Sources of Multiple Testing:}

\begin{enumerate}
\item Multiple metrics (10+ KPIs)
\item Multiple variants (A/B/C/D tests)
\item Multiple time points (peeking)
\item Multiple subgroups (age, country, etc.)
\item Multiple experiments running simultaneously
\end{enumerate}

\vspace{0.3cm}

\textbf{Correction Methods:}

\textbf{1. Bonferroni Correction:}

\[
\alpha_{\text{adj}} = \frac{\alpha}{m}
\]
where $m$ is number of tests.

\textbf{Example:} 10 metrics, want FWER = 0.05:
\[
\alpha_{\text{adj}} = 0.05 / 10 = 0.005
\]

\textbf{Pros:} Simple, controls FWER
\textbf{Cons:} Very conservative, low power
\end{column}

\begin{column}{0.5\textwidth}
\textbf{2. Holm-Bonferroni (Step-down):}

More powerful than Bonferroni:
\begin{enumerate}
\item Order p-values: $p_1 \leq p_2 \leq \cdots \leq p_m$
\item Compare $p_i$ to $\alpha/(m-i+1)$
\item Stop at first non-rejection
\end{enumerate}

\vspace{0.2cm}

\textbf{3. Benjamini-Hochberg (FDR control):}

Controls false discovery rate (less stringent):
\begin{enumerate}
\item Order p-values
\item Find largest $i$ where $p_i \leq \frac{i \cdot \alpha}{m}$
\item Reject $H_1, \ldots, H_i$
\end{enumerate}

\vspace{0.2cm}

\textbf{Python Implementation:}

\begin{verbatim}
from statsmodels.stats.multitest import
    multipletests

# P-values from 10 tests
p_values = [0.001, 0.02, 0.03, 0.04,
            0.06, 0.1, 0.2, 0.3, 0.5, 0.8]

# Bonferroni
reject_bonf, p_adj_bonf, _, _ =
    multipletests(p_values, alpha=0.05,
                  method='bonferroni')

# Benjamini-Hochberg
reject_bh, p_adj_bh, _, _ =
    multipletests(p_values, alpha=0.05,
                  method='fdr_bh')

print(f"Bonferroni: {sum(reject_bonf)}
      rejected")
print(f"BH: {sum(reject_bh)} rejected")
\end{verbatim}
\end{column}
\end{columns}
\end{frame}

% ================================================================
% KNOWLEDGE CHECK
% ================================================================

\begin{frame}{Knowledge Check: A/B Testing}
\textbf{Test your understanding}

\vspace{0.2cm}

\textbf{Question 1:} Your A/B test has p-value = 0.08. Your manager wants to ship variant B anyway because "it looks promising." What do you say?

\begin{enumerate}[A)]
\item Ship it, 0.08 is close enough to 0.05
\item \textcolor{forest}{\textbf{Don't ship - insufficient evidence. Risk of false positive is too high.}}
\item Collect more data until p < 0.05
\item Use Bayesian approach instead
\end{enumerate}

\pause

\begin{block}{Explanation}
\textcolor{forest}{\textbf{Answer: B}} - p = 0.08 means 8\% chance of seeing this result if there's no real effect. We set $\alpha = 0.05$ for a reason - to limit false positives. Shipping based on p = 0.08 means you're accepting 8\% Type I error rate. Option C (collect more data) is OK if pre-specified, but continuous peeking inflates error rates. Be disciplined about statistical thresholds!
\end{block}

\pause

\vspace{0.2cm}

\textbf{Question 2:} When should you use a multi-armed bandit instead of A/B test?

\begin{enumerate}[A)]
\item When you have limited traffic
\item When you need definitive statistical answer
\item \textcolor{forest}{\textbf{When minimizing regret matters more than statistical rigor}}
\item Never, A/B tests are always better
\end{enumerate}

\pause

\begin{block}{Explanation}
\textcolor{forest}{\textbf{Answer: C}} - Bandits adapt allocation to favor better variants, minimizing opportunity cost during experimentation. Use when: continuous optimization, many variants, direct revenue impact. Use A/B test when: one-time product decision, need clear winner for long-term commitment, statistical power critical for decision making.
\end{block}
\end{frame}

% ================================================================
% SUMMARY
% ================================================================

\begin{frame}{Summary \& Best Practices}
\textbf{Running successful experiments}

\begin{columns}
\begin{column}{0.5\textwidth}
\textbf{The Experimentation Checklist:}

\textbf{Before Experiment:}
\begin{enumerate}
\item Define clear hypothesis
\item Choose primary metric (one!)
\item Calculate required sample size
\item Set significance level ($\alpha = 0.05$)
\item Specify minimum detectable effect
\item Write analysis plan
\item Check randomization works
\end{enumerate}

\textbf{During Experiment:}
\begin{enumerate}
\item Monitor for bugs (not metrics!)
\item Check balance between groups
\item Don't peek at p-values
\item Run for pre-specified duration
\item Account for seasonality
\end{enumerate}

\textbf{After Experiment:}
\begin{enumerate}
\item Run pre-specified analysis
\item Check for metric movements
\item Segment analysis (exploratory)
\item Document everything
\item Make decision
\end{enumerate}
\end{column}

\begin{column}{0.5\textwidth}
\textbf{Common Mistakes:}

\begin{itemize}
\item[$\times$] Peeking and stopping early
\item[$\times$] Running too short (seasonality)
\item[$\times$] Not pre-registering analysis
\item[$\times$] Multiple testing without correction
\item[$\times$] Changing metric post-hoc
\item[$\times$] Insufficient sample size
\item[$\times$] Contamination between groups
\item[$\times$] Confusing statistical and practical significance
\end{itemize}

\vspace{0.3cm}

\textbf{Resources:}

\begin{itemize}
\item \textbf{Books:}
  \begin{itemize}
  \item "Trustworthy Online Controlled Experiments" (Kohavi et al.)
  \item "Experimental Design" (Montgomery)
  \end{itemize}

\item \textbf{Tools:}
  \begin{itemize}
  \item Statsmodels, SciPy (Python)
  \item GrowthBook, Optimizely, LaunchDarkly
  \end{itemize}

\item \textbf{Papers:}
  \begin{itemize}
  \item Peeking at A/B Tests (Johari et al.)
  \item CUPED (Deng et al., Microsoft)
  \end{itemize}
\end{itemize}

\begin{alertblock}{Golden Rule}
Pre-specify everything. Document everything. Don't p-hack!
\end{alertblock}
\end{column}
\end{columns}
\end{frame}

% ================================================================
% END OF A/B TESTING PRESENTATION
% ================================================================
