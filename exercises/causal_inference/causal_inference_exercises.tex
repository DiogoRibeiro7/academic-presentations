\documentclass[11pt,a4paper]{article}
\usepackage[utf8]{inputenc}
\usepackage[margin=1in]{geometry}
\usepackage{amsmath,amsthm,amssymb}
\usepackage{graphicx}
\usepackage{hyperref}
\usepackage{enumitem}
\usepackage{xcolor}
\usepackage{tikz}
\usetikzlibrary{shapes,arrows,positioning}
\usepackage{fancyhdr}

% Header and footer
\pagestyle{fancy}
\fancyhf{}
\lhead{Causal Inference - Problem Set}
\rhead{Diogo Ribeiro, ESMAD}
\cfoot{\thepage}

% Theorem environments
\newtheorem{exercise}{Exercise}
\newtheorem{problem}{Problem}

\title{\textbf{Causal Inference: Modern Methods}\\
Problem Set}
\author{Diogo Ribeiro\\
\textit{ESMAD - Escola Superior de Média Arte e Design}\\
\textit{Lead Data Scientist, Mysense.ai}\\
\href{mailto:dfr@esmad.ipp.pt}{dfr@esmad.ipp.pt}\\
ORCID: \href{https://orcid.org/0009-0001-2022-7072}{0009-0001-2022-7072}}
\date{\today}

\begin{document}

\maketitle

\begin{abstract}
This problem set covers modern methods in causal inference, including potential outcomes, directed acyclic graphs (DAGs), instrumental variables, regression discontinuity, and difference-in-differences. Students will gain practical experience identifying causal effects from observational data and understanding threats to causal identification.
\end{abstract}

\section{Potential Outcomes Framework}


\begin{exercise}[Fundamental Problem of Causal Inference]
Consider a job training program where $D_i = 1$ if individual $i$ receives training and $D_i = 0$ otherwise. Let $Y_i(1)$ be the potential outcome with training and $Y_i(0)$ without.

\begin{enumerate}[label=(\alph*)]
    \item Define the individual treatment effect (ITE), average treatment effect (ATE), and average treatment effect on the treated (ATT).

    \item Explain why we cannot observe $Y_i(1) - Y_i(0)$ for any individual.

    \item Show that the naive comparison $E[Y_i | D_i = 1] - E[Y_i | D_i = 0]$ equals:
    $$\underbrace{E[Y_i(1) - Y_i(0) | D_i = 1]}_{\text{ATT}} + \underbrace{E[Y_i(0) | D_i = 1] - E[Y_i(0) | D_i = 0]}_{\text{Selection Bias}}$$

    \item Under what conditions does the naive comparison identify the ATE?
\end{enumerate}
\end{exercise}

\begin{exercise}[Conditional Independence Assumption]
The conditional independence assumption (CIA) or unconfoundedness states:
$$\{Y_i(0), Y_i(1)\} \perp D_i | X_i$$

\begin{enumerate}[label=(\alph*)]
    \item Explain this assumption in plain language.

    \item Why is it called "unconfoundedness"?

    \item Give an example where CIA is plausible.

    \item Give an example where CIA is clearly violated.

    \item Under CIA, show that:
    $$E[Y_i(1) - Y_i(0) | X_i] = E[Y_i | D_i = 1, X_i] - E[Y_i | D_i = 0, X_i]$$
\end{enumerate}
\end{exercise}

\section{Directed Acyclic Graphs (DAGs)}

\begin{exercise}[Drawing and Interpreting DAGs]
For each scenario, draw a DAG and identify the causal effect of interest:

\begin{enumerate}[label=(\alph*)]
    \item \textbf{Education and Earnings:}
    \begin{itemize}
        \item Treatment: Years of education ($D$)
        \item Outcome: Annual earnings ($Y$)
        \item Confounders: Ability ($A$), Family background ($F$)
        \item Mediator: Job type ($J$)
    \end{itemize}
    Draw the DAG and identify:
    \begin{itemize}
        \item All paths from $D$ to $Y$
        \item Which paths are backdoor paths?
        \item What variables should be controlled?
        \item What is the direct effect of $D$ on $Y$ (not through $J$)?
    \end{itemize}

    \item \textbf{Collider Bias:}
    Consider $D \to Y \leftarrow U$ where $U$ is unobserved.
    \begin{itemize}
        \item Explain why conditioning on $Y$ induces association between $D$ and $U$
        \item Provide a real-world example of collider bias
    \end{itemize}

    \item \textbf{M-Bias:}
    Draw a DAG with structure: $U_1 \to D$, $U_2 \to Y$, $U_1 \to M \leftarrow U_2$.
    \begin{itemize}
        \item Is there an open backdoor path from $D$ to $Y$?
        \item What happens if you control for $M$?
        \item When is controlling for $M$ harmful?
    \end{itemize}
\end{enumerate}
\end{exercise}

\begin{problem}[Backdoor Criterion]
Use Pearl's backdoor criterion to determine the minimal sufficient adjustment set.

\textbf{Scenario:} Consider the following DAG:

\begin{center}
\begin{tikzpicture}[node distance=2cm]
    \node (D) {$D$};
    \node (Y) [right of=D, xshift=2cm] {$Y$};
    \node (X1) [above of=D] {$X_1$};
    \node (X2) [above of=Y] {$X_2$};
    \node (X3) [below of=D] {$X_3$};
    \node (U) [above of=X1, xshift=1cm] {$U$};

    \draw[->] (D) -- (Y);
    \draw[->] (X1) -- (D);
    \draw[->] (X1) -- (Y);
    \draw[->] (X2) -- (Y);
    \draw[->] (X3) -- (D);
    \draw[->] (U) -- (X1);
    \draw[->] (U) -- (X2);
\end{tikzpicture}
\end{center}

\begin{enumerate}
    \item List all paths from $D$ to $Y$
    \item Identify backdoor paths
    \item Find all valid adjustment sets
    \item What is the minimal sufficient adjustment set?
    \item Can we identify the causal effect if $U$ is unobserved?
\end{enumerate}
\end{problem}

\section{Instrumental Variables}

\begin{problem}[Returns to Education]
Estimate the returns to education using quarter of birth as an instrument.

\textbf{Background:} Angrist \& Krueger (1991) used quarter of birth as an instrument for education, exploiting compulsory schooling laws.

\textbf{Dataset:} Simulate or use provided data with:
\begin{itemize}
    \item $Y$: Log annual earnings
    \item $D$: Years of education
    \item $Z$: Quarter of birth indicator (born in Q1)
    \item $X$: Controls (age, region, etc.)
\end{itemize}

\textbf{Tasks:}
\begin{enumerate}
    \item \textbf{Motivation:}
    \begin{itemize}
        \item Why is education endogenous?
        \item Draw a DAG showing the endogeneity problem
        \item Explain how quarter of birth could be a valid instrument
    \end{itemize}

    \item \textbf{Validity Checks:}
    \begin{itemize}
        \item Test relevance: Regress $D$ on $Z$ (first stage)
        \item Compute first-stage F-statistic
        \item Discuss exclusion restriction: Why might it fail?
    \end{itemize}

    \item \textbf{Estimation:}
    \begin{itemize}
        \item Estimate OLS (biased)
        \item Estimate 2SLS
        \item Compare coefficients
        \item Compute heteroskedasticity-robust standard errors
    \end{itemize}

    \item \textbf{Interpretation:}
    \begin{itemize}
        \item What is the local average treatment effect (LATE)?
        \item Who are the compliers?
        \item Why might IV estimate differ from OLS?
    \end{itemize}

    \item \textbf{Sensitivity:}
    \begin{itemize}
        \item Add/remove controls
        \item Test overidentification (if multiple instruments)
        \item Discuss weak instruments problem
    \end{itemize}
\end{enumerate}
\end{problem}

\begin{problem}[Weak Instruments]
Investigate the consequences of weak instruments.

\textbf{Simulation:}
\begin{enumerate}
    \item Generate data: $n = 500$, instrument strength $\gamma \in \{0.05, 0.1, 0.3, 0.5\}$
    \begin{align*}
    D &= \gamma Z + \alpha U + \epsilon_D\\
    Y &= \beta D + \alpha U + \epsilon_Y
    \end{align*}
    where $U$ is unobserved confounder.

    \item For each $\gamma$:
    \begin{itemize}
        \item Estimate 2SLS
        \item Compute first-stage F-statistic
        \item Repeat 1000 times (Monte Carlo)
    \end{itemize}

    \item Analyze:
    \begin{itemize}
        \item Bias of 2SLS estimator
        \item Coverage of 95\% confidence intervals
        \item How weak is "weak"? (F < 10 rule of thumb)
    \end{itemize}

    \item Propose solutions:
    \begin{itemize}
        \item Limited information maximum likelihood (LIML)
        \item Anderson-Rubin test
        \item Weak instrument robust inference
    \end{itemize}
\end{enumerate}
\end{problem}

\section{Regression Discontinuity Design}

\begin{problem}[Class Size and Student Achievement]
Use RDD to estimate the effect of class size on test scores.

\textbf{Setup:} Schools with enrollment above 40 students must split into two classes.

\textbf{Data:}
\begin{itemize}
    \item Running variable: School enrollment (around cutoff of 40)
    \item Treatment: Small class (enrollment $\leq$ 40) vs. large class
    \item Outcome: Average test score
\end{itemize}

\textbf{Tasks:}
\begin{enumerate}
    \item \textbf{Identification:}
    \begin{itemize}
        \item State the RDD identification assumptions
        \item Why is continuity of potential outcomes crucial?
        \item Draw a hypothetical RDD plot
    \end{itemize}

    \item \textbf{Validity Checks:}
    \begin{itemize}
        \item Density test: Check for manipulation of running variable (McCrary test)
        \item Covariate balance: Test continuity of pre-treatment covariates
        \item Placebo cutoffs: Test at other points
    \end{itemize}

    \item \textbf{Estimation:}
    \begin{itemize}
        \item Local linear regression (both sides of cutoff)
        \item Optimal bandwidth selection (Imbens-Kalyanaraman or MSE-optimal)
        \item Robust inference (bias-corrected CI)
    \end{itemize}

    \item \textbf{Sensitivity:}
    \begin{itemize}
        \item Bandwidth sensitivity plot
        \item Polynomial order (linear vs. quadratic)
        \item Donut RDD (exclude observations near cutoff)
    \end{itemize}

    \item \textbf{Interpretation:}
    \begin{itemize}
        \item Is this effect local or global?
        \item What is the relevant population?
        \item External validity concerns
    \end{itemize}
\end{enumerate}
\end{problem}

\begin{problem}[Fuzzy RDD]
Extend to fuzzy RDD where treatment uptake is imperfect.

\textbf{Scenario:} Assignment to small class at cutoff, but:
\begin{itemize}
    \item Some schools just above cutoff have small classes (non-compliance)
    \item Some schools just below have large classes
\end{itemize}

\textbf{Tasks:}
\begin{enumerate}
    \item Explain the difference between sharp and fuzzy RDD
    \item Show that fuzzy RDD can be seen as an IV problem
    \item Estimate both the "first stage" (effect on treatment) and "reduced form" (effect on outcome)
    \item Compute the fuzzy RDD estimand (ratio)
    \item Interpret as local average treatment effect (LATE)
\end{enumerate}
\end{problem}

\section{Difference-in-Differences}

\begin{problem}[Minimum Wage and Employment]
Replicate Card \& Krueger (1994) analysis of minimum wage effects.

\textbf{Setup:}
\begin{itemize}
    \item Treatment: New Jersey minimum wage increase (April 1992)
    \item Control: Pennsylvania (no increase)
    \item Outcome: Employment at fast-food restaurants
    \item Periods: Before (Feb 1992) and After (Nov 1992)
\end{itemize}

\textbf{Tasks:}
\begin{enumerate}
    \item \textbf{Identification:}
    \begin{itemize}
        \item State parallel trends assumption
        \item Why is it crucial?
        \item Draw hypothetical trend plots
    \end{itemize}

    \item \textbf{Estimation:}
    \begin{itemize}
        \item Compute 2x2 DiD estimator (manual calculation)
        \item Estimate regression specification:
        $$Y_{it} = \alpha + \beta \text{Treat}_i + \gamma \text{Post}_t + \delta (\text{Treat}_i \times \text{Post}_t) + \epsilon_{it}$$
        \item Interpret $\delta$ (the DiD estimator)
        \item Compute cluster-robust standard errors (at state level)
    \end{itemize}

    \item \textbf{Parallel Trends Testing:}
    \begin{itemize}
        \item Test for pre-treatment differential trends
        \item Event study plot (if multiple time periods available)
        \item Placebo tests using pre-treatment periods
    \end{itemize}

    \item \textbf{Extensions:}
    \begin{itemize}
        \item Add covariates
        \item Triple differences (add another control group)
        \item Heterogeneous treatment effects by restaurant characteristics
    \end{itemize}
\end{enumerate}
\end{problem}

\begin{problem}[Staggered Adoption]
Analyze staggered treatment adoption with recent methods.

\textbf{Setup:} Multiple states adopt minimum wage increases at different times.

\textbf{Problem:} Two-way fixed effects (TWFE) can give negative weights to some treatment effects (Goodman-Bacon decomposition).

\textbf{Tasks:}
\begin{enumerate}
    \item Simulate staggered adoption data with heterogeneous effects
    \item Estimate naive TWFE:
    $$Y_{it} = \alpha_i + \lambda_t + \delta D_{it} + \epsilon_{it}$$
    \item Show that TWFE can be biased with:
    \begin{itemize}
        \item Treatment effect heterogeneity
        \item Dynamic effects
    \end{itemize}
    \item Implement modern alternatives:
    \begin{itemize}
        \item Callaway \& Sant'Anna (2021)
        \item Sun \& Abraham (2021)
        \item Borusyak et al. (2021)
    \end{itemize}
    \item Compare all estimates and discuss
\end{enumerate}
\end{problem}

\section{Matching and Propensity Scores}

\begin{problem}[Propensity Score Matching]
Estimate treatment effects using propensity score methods.

\textbf{Scenario:} Effect of college degree on earnings.

\textbf{Data:}
\begin{itemize}
    \item Treatment: College degree (vs. high school)
    \item Outcome: Annual earnings
    \item Covariates: Age, gender, parents' education, test scores, etc.
\end{itemize}

\textbf{Tasks:}
\begin{enumerate}
    \item \textbf{Propensity Score Estimation:}
    \begin{itemize}
        \item Fit logistic regression: $P(D = 1 | X)$
        \item Check for common support
        \item Plot propensity score distributions by treatment group
    \end{itemize}

    \item \textbf{Matching Methods:}
    \begin{itemize}
        \item Nearest neighbor matching (1:1, 1:k)
        \item Caliper matching
        \item Kernel matching
    \end{itemize}

    \item \textbf{Balance Checking:}
    \begin{itemize}
        \item Standardized mean differences before/after matching
        \item Love plot
        \item Variance ratios
    \end{itemize}

    \item \textbf{Estimate ATT:}
    \begin{itemize}
        \item Using matched samples
        \item IPW (inverse probability weighting)
        \item Doubly robust estimation
    \end{itemize}

    \item \textbf{Sensitivity Analysis:}
    \begin{itemize}
        \item Rosenbaum bounds
        \item How strong would hidden bias need to be?
    \end{itemize}
\end{enumerate}
\end{problem}

\section{Synthetic Control Methods}

\begin{problem}[Policy Evaluation]
Use synthetic control method to evaluate a policy intervention.

\textbf{Example:} German reunification effect on West German GDP (Abadie et al. 2015).

\textbf{Tasks:}
\begin{enumerate}
    \item Explain the synthetic control idea
    \item Construct synthetic control as weighted average of donor units:
    $$\hat{Y}_{0t} = \sum_{j=2}^{J+1} w_j Y_{jt}$$
    where weights minimize pre-treatment fit
    \item Estimate treatment effect: $\hat{\tau}_t = Y_{1t} - \hat{Y}_{0t}$
    \item Placebo tests: Apply method to control units
    \item Inference: Permutation-based p-values
    \item Sensitivity: Leave-one-out analysis
\end{enumerate}
\end{problem}

\section{Submission Guidelines}

\subsection{Format}
\begin{itemize}
    \item Python (Jupyter notebook) or R (R Markdown)
    \item Include all code, results, and visualizations
    \item Clear explanations and interpretations
    \item Well-documented code
\end{itemize}

\subsection{Evaluation}
\begin{itemize}
    \item \textbf{Identification (30\%):} Understanding of causal assumptions
    \item \textbf{Implementation (30\%):} Correct estimation procedures
    \item \textbf{Interpretation (25\%):} Thoughtful discussion of results
    \item \textbf{Validity Checks (15\%):} Robustness and sensitivity analyses
\end{itemize}

\section{Resources}

\textbf{Code:} \texttt{code/causal\_inference/}\\
\textbf{Bibliography:} \texttt{bibliographies/causal\_inference\_references.bib}\\
\textbf{Key Papers:}
\begin{itemize}
    \item Angrist \& Pischke (2008). \textit{Mostly Harmless Econometrics}
    \item Pearl (2009). \textit{Causality}
    \item Imbens \& Rubin (2015). \textit{Causal Inference for Statistics, Social, and Biomedical Sciences}
\end{itemize}

\vfill
\hrule
\vspace{0.5cm}
\noindent\textbf{Contact:} Diogo Ribeiro, dfr@esmad.ipp.pt

\end{document}
