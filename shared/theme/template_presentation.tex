\documentclass[aspectratio=169]{beamer}

% ================================================================
% ESMAD Professional Academic Presentation Template
% ================================================================

% Use the ESMAD theme
\usepackage{./esmad_beamer_theme}

% Author information (these are the defaults, customize as needed)
\authorname{Diogo Ribeiro}
\authoremail{dfr@esmad.ipp.pt}
\authororcid{0009-0001-2022-7072}
\authorinstitution{ESMAD - Escola Superior de Média Arte e Design}
\authorcompany{Lead Data Scientist, Mysense.ai}

% Presentation information
\title{Template Presentation}
\subtitle{Demonstrating the ESMAD Beamer Theme}
\date{\today}

% ================================================================
% Document
% ================================================================

\begin{document}

% Title slide
\begin{frame}
  \titlepage
\end{frame}

% Table of contents
\tocslide

% ================================================================
\section{Introduction}
% ================================================================

\begin{frame}{Welcome}
  \begin{itemize}
    \item This template demonstrates the ESMAD Beamer Theme
    \item Designed for professional academic presentations
    \item Consistent styling across all materials
    \item Easy to use and customize
  \end{itemize}

  \vspace{1em}

  \begin{block}{Key Features}
    \begin{itemize}
      \item Professional color scheme
      \item Custom theorem and definition boxes
      \item Integrated author information with ORCID
      \item Mathematical notation helpers
    \end{itemize}
  \end{block}
\end{frame}

\begin{frame}{Theme Features}
  \begin{columns}
    \column{0.5\textwidth}
    \textbf{Visual Elements:}
    \begin{itemize}
      \item Clean, modern design
      \item Professional blue palette
      \item Consistent typography
      \item High-quality figures
    \end{itemize}

    \column{0.5\textwidth}
    \textbf{Content Support:}
    \begin{itemize}
      \item Mathematical notation
      \item Code listings
      \item Algorithms
      \item References integration
    \end{itemize}
  \end{columns}
\end{frame}

% ================================================================
\section{Custom Environments}
% ================================================================

\begin{frame}{Theorem Box}
  \begin{theorembox}{Central Limit Theorem}
    Let $X_1, X_2, \ldots, X_n$ be i.i.d. random variables with mean $\mu$ and variance $\sigma^2$. Then:
    $$\frac{\bar{X}_n - \mu}{\sigma/\sqrt{n}} \xrightarrow{d} \Normal(0, 1)$$
    as $n \to \infty$.
  \end{theorembox}

  \vspace{0.5em}

  \textbf{Implications:}
  \begin{itemize}
    \item Sample means are approximately normally distributed
    \item Foundation for many statistical tests
    \item Justifies asymptotic approximations
  \end{itemize}
\end{frame}

\begin{frame}{Definition Box}
  \begin{definitionbox}{Definition: Markov Chain}
    A \textbf{Markov chain} is a sequence of random variables $X_0, X_1, X_2, \ldots$ satisfying the Markov property:
    $$\Prob(X_{n+1} = x | X_n, X_{n-1}, \ldots, X_0) = \Prob(X_{n+1} = x | X_n)$$
  \end{definitionbox}

  \vspace{0.5em}

  \begin{definitionbox}{Stationary Distribution}
    A distribution $\pi$ is \textbf{stationary} if:
    $$\pi(x) = \sum_{x'} \pi(x') P(x' \to x)$$
    where $P(x' \to x)$ is the transition probability.
  \end{definitionbox}
\end{frame}

\begin{frame}{Example Box}
  \begin{examplebox}{Example: Metropolis-Hastings}
    To sample from $\pi(x) = \Normal(0, 1)$:

    \begin{enumerate}
      \item Start with initial state $x_0 = 0$
      \item Propose $x' \sim \Normal(x_t, \sigma^2)$
      \item Compute acceptance probability:
      $$\alpha = \min\left(1, \frac{\pi(x')}{\pi(x_t)}\right)$$
      \item Accept $x_{t+1} = x'$ with probability $\alpha$, else $x_{t+1} = x_t$
    \end{enumerate}
  \end{examplebox}

  For $\sigma = 1$, this achieves acceptance rate $\approx$ 0.44 (optimal for 1D).
\end{frame}

\begin{frame}{Alert Box}
  \begin{alertbox}{Important: Convergence Diagnostics}
    Always check MCMC convergence before using samples for inference!

    \begin{itemize}
      \item \textbf{Trace plots:} Visual inspection of mixing
      \item \textbf{$\hat{R}$ statistic:} Should be $< 1.01$
      \item \textbf{Effective sample size:} Check for adequate samples
      \item \textbf{Autocorrelation:} Should decay quickly
    \end{itemize}
  \end{alertbox}

  \vspace{0.5em}

  \alert{Never trust results from poorly-converged chains!}
\end{frame}

% ================================================================
\section{Mathematical Content}
% ================================================================

\begin{frame}{Mathematical Notation}
  Using predefined commands from the theme:

  \begin{itemize}
    \item \textbf{Distributions:} $X \sim \Normal(\mu, \sigma^2)$, $Y \sim \Uniform(0, 1)$
    \item \textbf{Expectations:} $\E[X]$, $\Var(X)$, $\Cov(X, Y)$
    \item \textbf{Probability:} $\Prob(A \cap B)$
    \item \textbf{Operators:} $\argmax_{x} f(x)$, $\argmin_{\theta} L(\theta)$
    \item \textbf{Independence:} $X \indep Y | Z$
  \end{itemize}

  \vspace{1em}

  \begin{equation}
    p(\theta | y) = \frac{p(y | \theta) p(\theta)}{\int p(y | \theta') p(\theta') d\theta'}
  \end{equation}
\end{frame}

\begin{frame}{Equations and Algorithms}
  \begin{columns}
    \column{0.5\textwidth}
    \textbf{Bayesian Update:}
    \begin{align*}
      \text{Prior:} && p(\theta) \\
      \text{Likelihood:} && p(y | \theta) \\
      \text{Posterior:} && p(\theta | y)
    \end{align*}

    \column{0.5\textwidth}
    \textbf{Posterior Mean:}
    \begin{equation*}
      \hat{\theta} = \E[\theta | y] = \int \theta \, p(\theta | y) \, d\theta
    \end{equation*}
  \end{columns}

  \vspace{1em}

  \begin{block}{Matrix Notation}
    Linear model: $\vect{y} = \mat{X}\vect{\beta} + \vect{\epsilon}$

    OLS estimator: $\hat{\vect{\beta}} = (\mat{X}^T\mat{X})^{-1}\mat{X}^T\vect{y}$
  \end{block}
\end{frame}

% ================================================================
\section{Code and Algorithms}
% ================================================================

\begin{frame}[fragile]{Python Code}
  \begin{lstlisting}[language=Python, caption=Metropolis-Hastings Implementation]
import numpy as np

def metropolis_hastings(target, x0, n_samples, sigma=1.0):
    """Basic Metropolis-Hastings sampler."""
    samples = [x0]
    accepted = 0

    for i in range(n_samples):
        x_current = samples[-1]
        x_proposed = x_current + np.random.normal(0, sigma)

        # Acceptance probability
        alpha = min(1, target(x_proposed) / target(x_current))

        if np.random.rand() < alpha:
            samples.append(x_proposed)
            accepted += 1
        else:
            samples.append(x_current)

    return np.array(samples), accepted / n_samples
  \end{lstlisting}
\end{frame}

\begin{frame}[fragile]{R Code}
  \begin{lstlisting}[language=R, caption=Linear Regression with Diagnostics]
# Fit model
model <- lm(y ~ x1 + x2 + x3, data = df)

# Summary statistics
summary(model)

# Robust standard errors
library(sandwich)
library(lmtest)
coeftest(model, vcov = vcovHC(model, type = "HC1"))

# Diagnostics
par(mfrow = c(2, 2))
plot(model)
  \end{lstlisting}
\end{frame}

% ================================================================
\section{Figures and Results}
% ================================================================

\begin{frame}{Placeholder for Figures}
  \begin{figure}
    \centering
    % Example: \includegraphics[width=0.7\textwidth]{figures/trace_plot.pdf}
    \textcolor{ESMADGray}{[Insert trace plot here]}
    \caption{MCMC trace plot showing good mixing}
  \end{figure}

  \begin{itemize}
    \item Use vector graphics (PDF) when possible
    \item Ensure high resolution (at least 300 DPI for raster images)
    \item Keep color schemes consistent with theme
  \end{itemize}
\end{frame}

\begin{frame}{Tables}
  \begin{table}
    \centering
    \caption{Estimation Results Comparison}
    \begin{tabular}{lrrr}
      \toprule
      Method & Estimate & Std. Error & 95\% CI \\
      \midrule
      OLS & 1.234 & 0.056 & [1.124, 1.344] \\
      IV/2SLS & 1.456 & 0.089 & [1.281, 1.631] \\
      RDD & 1.389 & 0.102 & [1.189, 1.589] \\
      DiD & 1.412 & 0.078 & [1.259, 1.565] \\
      \bottomrule
    \end{tabular}
  \end{table}

  \textbf{Note:} Use \texttt{booktabs} package for professional tables.
\end{frame}

% ================================================================
\section{Best Practices}
% ================================================================

\begin{frame}{Presentation Tips}
  \begin{columns}
    \column{0.5\textwidth}
    \textbf{Content:}
    \begin{itemize}
      \item One main idea per slide
      \item Maximum 6-7 bullet points
      \item Use examples liberally
      \item Include intuition, not just math
    \end{itemize}

    \column{0.5\textwidth}
    \textbf{Visuals:}
    \begin{itemize}
      \item High-quality figures
      \item Consistent color scheme
      \item Clear axis labels
      \item Large enough fonts
    \end{itemize}
  \end{columns}

  \vspace{1em}

  \begin{alertbox}{Remember}
    Your slides support your talk, they don't replace it!
  \end{alertbox}
\end{frame}

\begin{frame}{Academic Integrity}
  \textbf{Always cite your sources:}

  \begin{itemize}
    \item Key papers and methods
    \item Figures and data from others
    \item Code and software packages
    \item Collaborators and funding
  \end{itemize}

  \vspace{0.5em}

  \textbf{Available bibliographies in this repository:}
  \begin{itemize}
    \item \texttt{shared/bibliographies/mcmc\_references.bib}
    \item \texttt{shared/bibliographies/causal\_inference\_references.bib}
    \item \texttt{shared/bibliographies/statistical\_learning\_references.bib}
  \end{itemize}
\end{frame}

% ================================================================
\section{Conclusion}
% ================================================================

\begin{frame}{Summary}
  \begin{itemize}
    \item The ESMAD Beamer Theme provides professional, consistent styling
    \item Custom environments for theorems, definitions, examples
    \item Built-in support for mathematical notation and code
    \item Easy to customize and extend
    \item Includes author information with ORCID
  \end{itemize}

  \vspace{1em}

  \begin{block}{Get Started}
    \begin{enumerate}
      \item Copy \texttt{theme/template\_presentation.tex}
      \item Customize author information
      \item Add your content
      \item Compile with \texttt{pdflatex} or \texttt{latexmk}
    \end{enumerate}
  \end{block}
\end{frame}

% Acknowledgments slide
\acknowledgmentsslide{
  \item ESMAD for institutional support
  \item Mysense.ai for industry collaboration
  \item Students and colleagues for feedback
  \item Open source community for LaTeX and Beamer
}

% Contact information slide
\contactslide

\end{document}
