\documentclass[aspectratio=169]{beamer}

\usepackage[utf8]{inputenc}
\usepackage[T1]{fontenc}
\usepackage{lmodern}
\usepackage{amsmath, amssymb}
\usepackage{graphicx}
\usepackage{booktabs}
\usepackage{hyperref}
\usepackage{listings}
\usepackage{mathtools}

\usetheme{Madrid}
\usecolortheme{default}

\lstset{
  basicstyle=\ttfamily\small,
  columns=fullflexible,
  keepspaces=true,
  breaklines=true,
  frame=single
}

%-------------------------------------------------------------------------------
% Metadata
%-------------------------------------------------------------------------------
\title[R: A Practical Introduction]{R: A Practical Introduction}
\author{}
\institute{}
\date{}

% Show section outline at the start of each section
\AtBeginSection{
  \begin{frame}{Outline}
    \tableofcontents[currentsection]
  \end{frame}
}

%-------------------------------------------------------------------------------
\begin{document}

\begin{frame}
  \titlepage
\end{frame}

\begin{frame}{Outline}
  \tableofcontents
\end{frame}

%===============================================================================
\section{Orientation}
%===============================================================================

\begin{frame}{What is R?}
  \begin{itemize}
    \item A language and environment for data analysis, statistics, and graphics.
    \item Combines interactive exploration with reproducible scripting.
    \item Runs on Linux, macOS, and Windows.
  \end{itemize}
\end{frame}

\begin{frame}{Why use R?}
  \begin{itemize}
    \item Rich statistical toolbox and graphics.
    \item First-class vectorisation and concise syntax.
    \item Vast ecosystem of packages for specialised tasks.
  \end{itemize}
\end{frame}

\begin{frame}{Session Basics}
  \begin{itemize}
    \item Prompt \texttt{>} evaluates expressions; results print unless assigned.
    \item Workspace holds objects; history records commands.
    \item Use scripts (\texttt{.R} files) for reproducibility.
  \end{itemize}
\end{frame}

%===============================================================================
\section{Vectors and Core Types}
%===============================================================================

\begin{frame}[fragile]{Numeric Vectors}
\begin{lstlisting}
x <- c(10.4, 5.6, 3.1, 6.4, 21.7)
2 * x + 1
log(x)
\end{lstlisting}
\end{frame}

\begin{frame}[fragile]{Sequences and Replication}
\begin{lstlisting}
1:10
seq(-5, 5, by = 0.5)
rep(1:3, each = 2)
\end{lstlisting}
\end{frame}

\begin{frame}[fragile]{Logical Values and Missing Data}
\begin{lstlisting}
x > 10
is.na(x)
x[is.na(x)] <- 0
\end{lstlisting}
\end{frame}

\begin{frame}[fragile]{Characters and Names}
\begin{lstlisting}
names <- c("Alice", "Bob", "Carol")
labs <- paste("X", 1:5, sep = "")
\end{lstlisting}
\end{frame}

\begin{frame}[fragile]{Indexing and Subsetting}
\begin{lstlisting}
x[1]       # first element
x[2:4]     # slice
x[-1]      # drop
x[x > 10]  # logical mask
\end{lstlisting}
\end{frame}

\begin{frame}{Hands-on \#1 (10 min)}
  \begin{itemize}
    \item Create a length-12 vector of random normals.
    \item Compute mean, median, sd, and a z-scored version.
    \item Replace any values $|z|>2$ by \texttt{NA}.
  \end{itemize}
\end{frame}

%===============================================================================
\section{Structured Data}
%===============================================================================

\begin{frame}[fragile]{Matrices and Arrays}
\begin{lstlisting}
A <- matrix(1:6, nrow = 2, ncol = 3)
A[1, 2]; A[, 2]; A[1, ]
\end{lstlisting}
\end{frame}

\begin{frame}[fragile]{Factors}
\begin{lstlisting}
sex <- factor(c("M", "F", "F", "M"))
levels(sex)
\end{lstlisting}
\end{frame}

\begin{frame}[fragile]{Lists}
\begin{lstlisting}
person <- list(name="Alice", age=30, scores=c(10,9,8))
person$name
person[["scores"]]
\end{lstlisting}
\end{frame}

\begin{frame}[fragile]{Data Frames}
\begin{lstlisting}
df <- data.frame(
  height=c(160,170,175),
  weight=c(55,65,72),
  sex=factor(c("F","M","M"))
)
str(df); summary(df)
\end{lstlisting}
\end{frame}

\begin{frame}{Hands-on \#2 (10 min)}
  \begin{itemize}
    \item Build a small data frame with 4 numeric columns and 1 factor.
    \item Add a derived column (e.g., BMI).
    \item Summarise each column and inspect structure.
  \end{itemize}
\end{frame}

%===============================================================================
\section{I/O and Data Management}
%===============================================================================

\begin{frame}[fragile]{Reading and Writing}
\begin{lstlisting}
dat <- read.csv("mydata.csv")
write.csv(dat, "out.csv", row.names = FALSE)
\end{lstlisting}
\end{frame}

\begin{frame}[fragile]{Workspaces and Scripts}
\begin{lstlisting}
save.image("session.RData")
load("session.RData")
source("analysis.R")
\end{lstlisting}
\end{frame}

\begin{frame}{Hands-on \#3 (5--10 min)}
  \begin{itemize}
    \item Load a built-in dataset (e.g., \texttt{iris}).
    \item Create a filtered subset and export to CSV.
  \end{itemize}
\end{frame}

%===============================================================================
\section{Control and Functions}
%===============================================================================

\begin{frame}[fragile]{Control Structures}
\begin{lstlisting}
if (x > 0) msg <- "positive" else msg <- "non-positive"
for (i in 1:3) print(i^2)
\end{lstlisting}
\end{frame}

\begin{frame}[fragile]{Functions}
\begin{lstlisting}
summ <- function(x, na.rm = FALSE) {
  if (!is.numeric(x)) stop("x must be numeric")
  if (na.rm) x <- x[!is.na(x)]
  list(mean=mean(x), median=median(x), sd=sd(x))
}
\end{lstlisting}
\end{frame}

\begin{frame}[fragile]{... and Argument Passing}
\begin{lstlisting}
my_plot <- function(x, y, ...) {
  plot(x, y, ...)
}
\end{lstlisting}
\end{frame}

\begin{frame}{Hands-on \#4 (10 min)}
  \begin{itemize}
    \item Write a function that standardises a numeric vector and clips to $[-3,3]$.
    \item Add input checks and an \texttt{na.rm} option.
  \end{itemize}
\end{frame}

%===============================================================================
\section{Statistical Modelling}
%===============================================================================

\begin{frame}[fragile]{Formula Interface}
\begin{lstlisting}
response ~ x1 + x2
y ~ x1 * x2       # main effects + interaction
\end{lstlisting}
\end{frame}

\begin{frame}[fragile]{Linear Models}
\begin{lstlisting}
fit <- lm(Sepal.Length ~ Sepal.Width + Species, data = iris)
summary(fit)
coef(fit); resid(fit); fitted(fit)
\end{lstlisting}
\end{frame}

\begin{frame}[fragile]{Model Diagnostics and Comparison}
\begin{lstlisting}
par(mfrow=c(2,2)); plot(fit)
fit1 <- lm(y ~ x1, data=df)
fit2 <- lm(y ~ x1 + x2, data=df)
anova(fit1, fit2)
\end{lstlisting}
\end{frame}

\begin{frame}[fragile]{Generalised Linear Models}
\begin{lstlisting}
glm_fit <- glm(y ~ x1 + x2, family = binomial, data = df)
summary(glm_fit)
\end{lstlisting}
\end{frame}

\begin{frame}{Hands-on \#5 (10 min)}
  \begin{itemize}
    \item Fit a linear model on a tabular dataset.
    \item Inspect coefficients, residuals, and diagnostic plots.
    \item Compare nested models with \texttt{anova()}.
  \end{itemize}
\end{frame}

%===============================================================================
\section{Graphics}
%===============================================================================

\begin{frame}[fragile]{High-Level Graphics}
\begin{lstlisting}
plot(iris$Sepal.Length, iris$Sepal.Width)
hist(iris$Sepal.Length)
boxplot(Sepal.Length ~ Species, data=iris)
\end{lstlisting}
\end{frame}

\begin{frame}[fragile]{Low-Level Additions and Devices}
\begin{lstlisting}
plot(iris$Sepal.Length, iris$Sepal.Width)
abline(lm(Sepal.Width ~ Sepal.Length, data=iris))
pdf("plot.pdf"); plot(1:10); dev.off()
\end{lstlisting}
\end{frame}

\begin{frame}{Hands-on \#6 (5--10 min)}
  \begin{itemize}
    \item Produce a scatter plot with a fitted line and labels.
    \item Export to PDF and PNG.
  \end{itemize}
\end{frame}

%===============================================================================
\section{Packages and Workflow}
%===============================================================================

\begin{frame}[fragile]{Packages}
\begin{lstlisting}
install.packages("ggplot2")
library(ggplot2)
\end{lstlisting}
\end{frame}

\begin{frame}{Project-Oriented Workflow}
  \begin{itemize}
    \item One project per analysis; keep data, code, and outputs organised.
    \item Use version control to track changes.
    \item Prefer scripts over manual steps for repeatability.
  \end{itemize}
\end{frame}

%===============================================================================
\section{Wrap-up}
%===============================================================================

\begin{frame}{Putting It Together}
  \begin{itemize}
    \item Core language: vectors, indexing, data structures.
    \item Data I/O and management.
    \item Functions, modelling, and graphics.
    \item Packages and a reproducible workflow.
  \end{itemize}
\end{frame}

\begin{frame}{Questions}
  \centering
  \Large Questions and discussion
\end{frame}

\end{document}
